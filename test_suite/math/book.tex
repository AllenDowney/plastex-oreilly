% LaTeX source for ``Think Stats:
% Probability and Statistics for Programmers''
% Copyright 2011  Allen B. Downey.

% License: Creative Commons Attribution-NonCommercial 3.0 Unported License.
% http://creativecommons.org/licenses/by-nc/3.0/
%

%\documentclass[10pt,b5paper]{book}
\documentclass[12pt]{book}
\usepackage[width=5.5in,height=8.5in,
  hmarginratio=3:2,vmarginratio=1:1]{geometry}

% for some of these packages, you might have to install
% texlive-latex-extra (in Ubuntu)

\usepackage[T1]{fontenc}
\usepackage{textcomp}
\usepackage{mathpazo}

%\usepackage{pslatex}

\usepackage{url}
\usepackage{fancyhdr}
\usepackage{graphicx}
\usepackage{subfig}
\usepackage{amsmath}
\usepackage{amsthm}
%\usepackage{amssymb}
\usepackage{makeidx}
\usepackage{setspace}
\usepackage{hevea}                           
\usepackage{upquote}


\title{Think Stats}
\author{Allen B. Downey}

\newcommand{\thetitle}{Example file for testing MathML}

% these styles get translated in CSS for the HTML version
\newstyle{a:link}{color:black;}
\newstyle{p+p}{margin-top:1em;margin-bottom:1em}
\newstyle{img}{border:0px}

% change the arrows in the HTML version
\setlinkstext
  {\imgsrc[ALT="Previous"]{back.png}}
  {\imgsrc[ALT="Up"]{up.png}}
  {\imgsrc[ALT="Next"]{next.png}} 

\makeindex

\newif\ifplastex
\plastexfalse

\begin{document}

\frontmatter

\ifplastex
    \usepackage{localdef}
    \maketitle

\else


\fi

\chapter{Math examples}

Display math:

\[ \mu = \frac{1}{n} \sum_i x_i \]

Another display equation:

\[ \sigma^2 = \frac{1}{n} \sum_i (x_i - \mu)^2 \]

The term $x_i - \mu$ is called the ``deviation from the mean,'' so
variance is the mean squared deviation, which is why it is denoted
$\sigma^2$.  The square root of variance, $\sigma$, is called the {\bf
  standard deviation}.

Here's a math expression that should get translated to a mathphrase: 
$a^2 + b^2 = c^2$.

Here's a math expression that has super and subscripts on the same
character, so it should be MathML: $a_i^2$.

Here's a math expression that contains a symbol with no unicode in Math.py,
so it should get translated to MathML: $a \leadsto b$.

\end{document}

