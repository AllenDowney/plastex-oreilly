

\makeatletter
\def\@makeschapterhead#1{%
  \vbox to 68pt{\vskip30pt\vskip-\headheight\vskip\headsep\vskip-\topskip\vskip-28pt\parindent\z@ \leftskip0pt plus1fill
{{\CTfont  #1\par}}%
    \vfill
   }}
\makeatother

\setcounter{page}{13}
\chapter*{Preface}{}{}

\Fint{This book grew out of my experience of working with data for various
companies in the tech} industry. It is a collection of those concepts
and techniques that I have found to be the most useful, including many
topics that I wish I had known earlier---but didn't.

My degree is in physics, but I also worked as a software engineer for
several years. The book reflects this dual heritage. On the one hand,
it is written for programmers and others in the software field: I
assume that you, like me, have the ability to write your own programs
to manipulate data in any way you want.

On the other hand, the way I think about data has been shaped by
my background and education.  As a physicist, I am not content merely
to describe data or to make black-box predictions: the purpose of an
analysis is always to develop an understanding for the processes or
mechanisms that give rise to the data that we observe.

The instrument to express such understanding is the \emph{model}: \index{modeling!about} a
description of the system under study (in other words, not just a
description of the data!), simplified as necessary but nevertheless
capturing the relevant information. A model may be crude (``Assume a
spherical cow$\,\ldots$''), but if it helps us develop better insight on
how the system works, it is a successful model nevertheless.
(Additional precision can often be obtained at a later time, if it is
really necessary.)

This emphasis on models and simplified descriptions is not universal:
other authors and practitioners will make different choices. But it is
essential to my approach and point of view.

This is a rather personal book. Although I have tried to be reasonably
comprehensive, I have selected the topics that I consider relevant and
useful in practice---whether they are part of the ``canon'' or not.
Also included are several topics that you won't find in any other book
on data analysis. Although neither new nor original, they are usually
not used or discussed in this particular context---but I find them
indispensable.

Throughout the book, I freely offer specific, explicit advice,
opinions, and assessments. These remarks are reflections of my
personal interest, experience, and understanding. I do not claim that
my point of view is necessarily correct: evaluate what I say for
yourself and feel free to adapt it to your needs.  In my view, a
specific, well-argued position is of greater use than a sterile
laundry list of possible algorithms---even if you later decide to
disagree with me.  The value is not in the opinion but rather in the
arguments leading up to it. If your arguments are better than mine, or
even just more agreeable to you, then I will have achieved my purpose!

Data analysis, as I understand it, is not a fixed set of techniques.
It is a way of life, and it has a name: curiosity. There is always
something else to find out and something more to learn.  This book is
not the last word on the matter; it is merely a snapshot in time:
things I knew about and found useful today.

``Works are of value only if they give rise to better ones.''

(Alexander von Humboldt, writing to Charles Darwin, 18 September 1839)


% ============================================================
\section*{Before We Begin}

\index{data analysis!about} 

More data analysis efforts seem to go bad because of an excess of
sophistication rather than a lack of it.

This may come as a surprise, but it has been my experience again and
again. As a consultant, I am often called in when the initial project
team has already gotten stuck. Rarely (if ever) does the problem
turn out to be that the team did not have the required skills.  On the
contrary, I usually find that they tried to do something unnecessarily
complicated and are now struggling with the consequences of their own
invention! 

Based on what I have seen, two particular risk areas stand out:

\begin{itemize}
\item The use of ``statistical'' concepts that are only partially
  understood (and given the relative obscurity of most of statistics,
  this includes virtually \emph{all} statistical concepts)
\item Complicated (and expensive) black-box solutions when a simple
  and transparent approach would have worked at least as well or
  better
\end{itemize}

I strongly recommend that you make it a habit to avoid all statistical
language. Keep it simple and stick to what you know for sure.  There
is absolutely nothing wrong with speaking of the ``range over which
points spread,'' because this phrase means exactly what it says: the
range over which points spread, and only that! Once we start talking
about ``standard deviations,'' this clarity is gone.  Are we still
talking about the \emph{observed} width of the distribution?  Or are
we talking about one specific \emph{measure} for this width?  (The
standard deviation is only one of several that are available.)  Are we
already making an implicit \emph{assumption} about the nature of the
distribution?  (The standard deviation is only suitable under certain
conditions, which are often not fulfilled in practice.) Or are we even
confusing the \emph{predictions} we could make if these assumptions
were true with the actual data? (The moment someone talks about ``95
percent anything'' we know it's the latter!)

I'd also like to remind you not to discard simple methods until they
have been \emph{proven} insufficient. Simple solutions are frequently
rather effective: the marginal benefit that more complicated methods
can deliver is often quite small (and may be in no reasonable relation
to the increased cost). More importantly, simple methods have fewer
opportunities to go wrong or to obscure the obvious.

True story: a company was tracking the occurrence of defects over
time. Of course, the actual number of defects varied quite a bit from
one day to the next, and they were looking for a way to obtain an
estimate for the typical number of expected defects. The solution
proposed by their IT department involved a compute cluster running a
neural network! (I am not making this up.)  In fact, a one-line
calculation (involving a moving average or single exponential
smoothing) is all that was needed.

I think the primary reason for this tendency to make data analysis
projects more complicated than they are is \emph{discomfort}:
discomfort with an unfamiliar problem space and uncertainty about how
to proceed. This discomfort and uncertainty creates a desire to bring
in the ``big guns'': fancy terminology, heavy machinery, large
projects. In reality, of course, the opposite is true: the
complexities of the ``solution'' overwhelm the original problem, and
nothing gets accomplished.

Data analysis does not have to be all that hard. Although there are
situations when elementary methods will no longer be sufficient, they
are much less prevalent than you might expect. In the vast majority of
cases, curiosity and a healthy dose of common sense will serve you
well.

The attitude that I am trying to convey can be summarized in a few
points:

\begin{quote}
Simple is better than complex.

Cheap is better than expensive.

Explicit is better than opaque.

Purpose is more important than process.

Insight is more important than precision.

Understanding is more important than technique.

Think more, work less.
\end{quote}

Although I do acknowledge that the items on the right are necessary
at times, I will give preference to those on the left whenever
possible.

It is in this spirit that I am offering the concepts and techniques
that make up the rest of this book.


\section*{Conventions Used in This Book}

The following typographical conventions are used in this book:
\begin{unnumlist}
\subparagraph{Italic}
\item Indicates new terms, URLs, and email addresses
\subparagraph{\fontsize{8}{12}\tt Constant width}
\item Used to refer to language and script elements
\end{unnumlist}


\section{Using Code Examples}
This book is here to help you get your job done. In general, you may use the code in this book
in your programs and documentation. You do not need to contact us for permission unless
you�re reproducing a significant portion of the code. For example, writing a program that uses
several chunks of code from this book does not require permission. Selling or distributing a
CD-ROM of examples from O�Reilly books does require permission. Answering a question by
citing this book and quoting example code does not require permission. Incorporating a
significant amount of example code from this book into your product�s documentation does
require permission.


We appreciate, but do not require, attribution. An attribution usually includes the title, author,
publisher, and ISBN. For example: ``{\it Data Analysis with Open Source Tools}, by Philipp K. Janert. Copyright 2011 Philipp K. Janert,
 978-0-596-80235-6.''

If you feel your use of code examples falls outside fair use or the permission given above, feel
free to contact us at {\it permissions@oreilly.com.}\vspace*{-4pt}

\section*{Safari� Books Online}

\vbox{{\begin{tabular}{@{}l@{}}
\hspace*{30pt}{\lower-.2pt\hbox{\blackfiftyink\fontfamily{Myriad}\bsemiseries\fontsize{15}{17}\selectfont  .}\lower1pt\hbox{\fontfamily{Myriad}\bfseries\fontsize{9}{9}\selectfont>\blackink}}\\[-6pt]
{\fontfamily{Myriad}\bsemiseries\fontsize{15}{17}\selectfont Safari}\\[-9pt]
\hspace*{7.5pt}{\fontfamily{Myriad}\bsemiseries\fontsize{5}{9}\selectfont Books online}
\end{tabular}}}\vspace*{-21pt}

\vbox{\leftskip48pt Safari Books Online is an on-demand digital library that lets you easily search
over 7,500 technology and creative reference books and videos to find the
answers you need quickly.}

With a subscription, you can read any page and watch any video from our library online. Read
books on your cell phone and mobile devices. Access new titles before they are available for
print, and get exclusive access to manuscripts in development and post feedback for the
authors. Copy and paste code samples, organize your favorites, download chapters, bookmark
key sections, create notes, print out pages, and benefit from tons of other time-saving features.

O'Reilly Media has uploaded this book to the Safari Books Online service. To have full digital
access to this book and others on similar topics from O�Reilly and other publishers, sign up for
free at {\it http://my.safaribooksonline.com}.\vspace*{-4pt}

\section{How to Contact Us}
Please address comments and questions concerning this book to the publisher:

\begin{quote}
O'Reilly Media, Inc.

1005 Gravenstein Highway North

Sebastopol, CA 95472

800-998-9938 (in the United States or Canada)

707-829-0515 (international or local)

707-829-0104 (fax)
\end{quote}

We have a web page for this book, where we list errata, examples, and any additional
information. You can access this page at:

\begin{quote}
{\it http://oreilly.com/catalog/9780596802356}
\end{quote}


To comment or ask technical questions about this book, send email to:

\begin{quote}
{\it bookquestions@oreilly.com}
\end{quote}

For more information about our books, conferences, Resource Centers, and the O'Reilly
Network, see our website at:

\begin{quote}
{\it http://oreilly.com}
\end{quote}


% ============================================================
\section*{Acknowledgments}

It was a pleasure to work with O'Reilly on this project. In
particular, O'Reilly has been most accommodating with regard to the
technical challenges raised by my need to include (for an O'Reilly
book) an uncommonly large amount of mathematical material in the
manuscript.

Mike Loukides has accompanied this project as the editor since
its beginning. I have enjoyed our conversations about life, the
universe, and everything, and I appreciate his comments about the
manuscript---either way.

I'd like to thank several of my friends for their help in bringing
this book about:

\begin{itemize}
\item Elizabeth Robson, for making the connection
\item Austin King, for pointing out the obvious
\item Scott White, for suffering my questions gladly
\item Richard Kreckel, for much-needed advice
\end{itemize}

As always, special thanks go to PAUL Schrader (Bremen).

The manuscript benefited from the feedback I received from various
reviewers. Michael E.\ Driscoll, Zachary Kessin, and Austin King read
all or parts of the manuscript and provided valuable comments.

I enjoyed personal correspondence with Joseph Adler, Joe Darcy, Hilary
Mason, Stephen Weston, Scott White, and Brian Zimmer. All very
generously provided expert advice on specific topics.

Particular thanks go to Richard  Kreckel, who provided uncommonly
detailed and insightful feedback on most of the manuscript.

During the preparation of this book, the excellent collection at the
University of Washington libraries was an especially valuable resource
to me.\clearpage

Authors usually thank their spouses for their ``patience and support''
or words to that effect. Unless one has lived through the actual
experience, one cannot fully comprehend how true this is.  Over the
last three years, Angela has endured what must have seemed like a
nearly continuous stream of whining, frustration, and
desperation---punctuated by occasional outbursts of exhilaration and
grandiosity---all of which before the background of the self-centered
and self-absorbed attitude of a typical author. Her patience and
support were unfailing.  It's her turn now.
%
%
%\chapter*{About the Author}{}{}
%
%
%After previous careers in physics and software 
%development, {\bf Philipp K. Janert} currently 
%provides consulting services for data analysis, 
%algorithm development, and mathematical modeling. 
%He has worked for small start-ups and in large 
%corporate environments, both in the U.S. and 
%overseas. He prefers simple solutions that work 
%to complicated ones that don't, and thinks that 
%purpose is more important than process. Philipp 
%is the author of ``Gnuplot in Action:  Understanding
%Data with Graphs'' (Manning Publications), and has 
%written for the O'Reilly Network, IBM developerWorks, 
%and IEEE Software. He is named inventor on a handful 
%of patents, and is an occasional contributor to CPAN. 
%He holds a Ph.D. in theoretical physics from the 
%University of Washington. Visit his company website 
%at \url{www.principal-value.com.}
%
%
%%{\bf Philipp K. Janert}
%%is chief consultant at Principal Value, LLC. He has worked for small start-ups and in large corporate environments---including
%%Amazon.com---where he initiated and led several projects to improve order fulfillment processes. Philipp has written about software and
%%software development for the O'Reilly Network, IBM developerWorks, IEEE Software, and {\it Linux Magazine}.\vspace*{-6pt}
%
%%\leaflong{6pt}
%
%\section*{Colophon}
%The animal on the cover of {\it Data Analysis with Open Source Tools} is a common kite, most likely a member of the genus {\it Milvus}. Kites are
%medium-size raptors with long wings and forked tails. They are noted for their elegant, soaring flight. They are also called ``gledes''
%(for their gliding motion) and, like the flying toys, they appear to ride effortlessly on air currents.
%
%The genus {\it Milvus} is a group of Old World kites, including three or four species and numerous subspecies. These kites are opportunistic
%feeders who hunt small animals, such as birds, fish, rodents, and earthworms, and also eat carrion, including sheep and cow carcasses.
%They have been observed to steal prey from other birds. They may live 25 to 30~years in the wild.
%
%The genus dates to prehistoric times; an Israeli {\it Milvus pygmaeus} specimen is thought to be between 1.8 million and 780,000 years old.
%Biblical references to kites probably refer to birds of this genus. In {\it Coriolanus}, Shakespeare calls Rome ``the city of kites and crows,''
%commenting on the birds' prevalence in urban areas.
%
%The most widespread member of the genus is the black kite ({\it Milvus migrans}), found in Europe, Asia, Africa, and Australia. These kites are
%very common in many parts of their habitat and are well adapted to city life. Attracted by smoke, they sometimes hunt by capturing small
%animals fleeing from fires.
%
%The other notable member of {\it Milvus} is the red kite ({\it Milvus milvus}), which is slightly larger than the black kite and is distinguished by a
%rufous body and tail. Red kites are found only in Europe. They were very common in Britain until 1800, but the population was devastated
%by poisoning and habitat loss, and by 1930, fewer than 20 birds remained. Since then, kites have made a comeback in Wales and have been
%reintroduced elsewhere in Britain.
%
%The cover image is from Cassell's {\it Natural History}, Volume III. The cover font is Adobe ITC Garamond; the text font is Linotype Birka; the
%heading font is Adobe Myriad Condensed; and the code font is LucasFont's TheSansMonoCondensed.
%
%\clearpage
%
%\thispagestyle{empty}
%
%\cleardoublepage