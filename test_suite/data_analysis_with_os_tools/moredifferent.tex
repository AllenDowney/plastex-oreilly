
% spellcheck: ok
% mikes: ok

% ============================================================
\chapter{Intermezzo: When More Is Different}{}{}


%\vspace*{-24pt}
\Fint{When dealing with some of the more computationally intensive data
analysis or mining} algorithms, you may encounter an unexpected
obstacle: \emph{the brick wall}. Programs or algorithms that seemed to
work just fine turn out not to work once in production. And I don't
mean that they work slower than expected. I mean they do not work at
all!

\enlargethispage{12pt}

Of course, performance \index{performance!matrix operations and other computational applications|(} and scalability problems are familiar to most
enterprise developers. However, the kinds of problems that arise in
data-centric or computationally intensive applications are different,
and most enterprise programmers (and, in fact, most computer science
graduates) are badly prepared for them.

Let's try an example: Table \ref{tbl:moredifferent} shows the time
required to perform 10 matrix multiplications for square matrices of
various size. \index{matrix operations} (The details of matrix multiplication don't concern us
here; suffice it to say that it's the basic operation in almost all
problems involving matrices and is at the heart of operator
decomposition problems, including the principal component analysis
introduced in Chapter \ref{ch:reduction}.)

% How long will it take to multiply two $10000 \times 10000$ matrices?
\begin{table}[h]
\vspace*{-15pt}
\def\a{\hphantom{0}}
\def\b{\hphantom{0,}}
\def\vrl{\smash{\vrule height64pt  width.25pt depth5pt}}
\tabcolsep34pt
\tbl{Time required to perform 10 matrix multiplications for
  square matrices of different sizes\label{tbl:moredifferent}}{%
\begin{tabular}{@{\hskip12pt}c@{\hskip12pt}c@{\hskip12pt}c@{\hskip12pt}}
\toprule
  \TCH{Size {\itshape n}} && \TCH{Time [seconds]}\\
\colrule
   \b100     &&   \a\a0.00 \\
   \b200     &&   \a\a0.06 \\
   \b500     &&   \a\a2.12 \\
 1,000     &&  \a22.44 \\
 2,000     &\vrl& 176.22 \\\botrule
\end{tabular}}\vspace*{-12pt}
\end{table}\pagebreak

Would you agree that the data in Table \ref{tbl:moredifferent} does
not look too threatening? For a $\text{2,000} \times \text{2,000}$
matrix, the time required is a shade under three minutes. How long
might it take to perform the same operation for a $\text{10,000}
\times \text{10,000}$ matrix? Five, maybe ten minutes? Yeah, right. It
takes \emph{five hours}! And if you need to go a little bit bigger
still---say, $\text{30,000} \times \text{30,000}$, the computation
will take five \emph{days}.

What we observe here is typical of many computationally intensive
algorithms: they consume disproportionately more time as the problem
size becomes larger. Of course, we have all heard about this in school,
but our intuition for the reality of this phenomenon is usually not
very good. Even if we run a few tests on small data sets, we fail to
spot the trouble: sure, the program takes longer as the data sets get
larger, but it all seems quite reasonable. Nevertheless, we tend to be
unprepared for what appears to be a huge \emph{jump} in the required
time as we increase the data set by a seemingly not very large factor.
(Remember: what took us from three minutes to five hours was an
increase in the problem size by a factor of 5---not even an order
of magnitude!)

The problem is that, unless you have explicitly worked on either a
numerical or a combinatorial problem in the past, you probably have
never encountered the kind of scaling behavior exhibited by
computational or combinatorial problems. This skews our perception.

Where are you most likely to encounter perceptible performance
problems in an enterprise environment? Answer: slow database queries!
We all have encountered the frustration resulting from queries that
perform a full table scan instead of using an indexed lookup
(regardless whether no index is available or the query optimizer fails
to use it). Yet a query that performs a full table scan rather than
using an index exhibits one of the most benign forms of scaling: from
$\mathcal{O}(\log n)$ (meaning that the response time is largely
insensitive to the size of the table) to $\mathcal{O}(n)$ (meaning
that doubling the table size will double the response time).

In contrast, matrix operations---such as the matrix multiplication
encountered in the earlier example---scale as $\mathcal{O}(n^3)$; this
means that if the problem doubles in size, then the time required
grows by a factor of \emph{8} (because $2^3 = 8$). In other words,
as you go from a $\text{2,000} \times \text{2,000}$ matrix to a
$\text{4,000} \times \text{4,000}$ matrix, the problem will take
almost 10 times as long; and if you go to a $\text{10,000} \times
\text{10,000}$ matrix, it will take $5^3 = 125$ times as long. Oops.

And this is the good news. Many combinatorial problems (such as the
Traveling Salesman problem and similar problems) don't scale according
to a power law (such as $\mathcal{O}(n^3)$) but instead scale
exponentially ($\mathcal{O}(e^n)$). In these cases, you will hit the
brick wall \emph{much} faster and much more brutally. For such
problems, an incremental increase in the size of the problem (\ie,
from $n$ to $n+1$) will typically at least \emph{double} the runtime.
In other words, the last element to calculate takes as much time as
all the previous elements \emph{taken together}. System sizes of
around $n=50$ are frequently the end of the line.  With extreme effort
you might be able to push it to $n=55$, but $n=100$ will be entirely
out of reach.

The reason I stress this kind of problem so much is that in my
experience, not only are most enterprise developers unprepared for the
reality of it but also that the standard set of software engineering
practices and attitudes is entirely inadequate to deal with them.  I
once heard a programmer say, ``It's all just engineering'' in response
to challenges about the likely performance problems of a computational
system he was working on. Nothing could be further from the truth: no
amount of low-level performance tuning will save a program of this
nature that is algorithmically hosed---and no amount of faster
hardware, either. Moreover, ``standard software engineering
practices'' are either of no help or are even entirely inapplicable
(we'll see an example in a moment).

Most disturbing to me was his casual, almost blissful ignorance---this
coming from a guy who definitely \emph{should} have known better.

% ============================================================
\section{A Horror Story}

I was once called into a project in its thirteenth hour---they had far
exceeded both their budget and their schedule and were about to be
shut down for good because they could not make their system work.
They had been trying to build an internal tool that was intended to
solve what was, essentially, a combinatorial problem.  The tool was
supposed to be used interactively: the user supplies some inputs and
receives an answer within, at most, a few minutes. By the time I got
involved, the team had labored for over a year, but the minimum
response time achieved by their system exceeded \emph{12
  hours}---even though it ran on a very expensive (and very expensive
to operate) supercomputer.

After a couple of weeks, I came up with an improved algorithm that
calculated answers in real time and could run on a laptop.

No amount of ``engineering'' will be able to deliver that kind of
speed-up.

How was this possible? By attacking the problem on many different
levels.  First of all, we made sure we fully \emph{understood the
  problem domain}. The original project team had always been a little
vague about what exactly the program was trying to calculate, as a
result their ``domain model'' was not truly logically consistent.
Hence the first thing to do was to put the whole problem on sound
mathematical footing.  Second, we \emph{redefined the problem}: the
original program had attempted to calculate a certain quantity by
explicit enumeration of all possible combinations, whereas the new
solution calculated an approximation instead. This was warranted
because the input data was not known very precisely, anyway, and
because we were able to show that the uncertainty introduced by the
approximation was less than the uncertainty already present in the
data. Third, we \emph{treated hot spots differently than the happy
  case}: the new algorithm could calculate the result to higher
accuracy, but it did so only when the added accuracy was needed.
Fourth, we used efficient data structures and implemented some core
pieces ourselves instead of relying on general-purpose libraries; we
also judiciously precalculated and cached some frequently used
intermediate results.

After putting the whole effort on a conceptually consistent footing,
the most important contribution was changing the problem
definition: dropping the exact approach, which was unnecessary and
infeasible, and adopting an approximate solution that was cheap and
all that was required.

% ============================================================
\section{Some Suggestions}

Computational and combinatorial programming is really different. It
runs into different limits and requires different techniques. Most
important is the appropriate choice of algorithm at the outset, since
no amount of low-level tuning or ``engineering'' will save a program
that is algorithmically flawed.

Here is a list of recommendations in case you find yourself setting
out on a project that involves heavy computation or deals with
combinatorial complexity issues:

%\begin{unnumlist}
\paragraph{Do your homework.}
Understand computational complexity, know the complexity of the
algorithm you intend to use, and research the different algorithms
(and their trade-offs) available for your kind of problem. Read
broadly---although the exact problem as specified may turn out to be
intractable, you may find that a small change in the requirements may
lead to a much simpler problem. It is definitely worth it to renegotiate
the problem with the customer or end users than setting out on a
project that is infeasible from the outset. (Skiena's \emph{Algorithm
  Design Manual} is a particularly good resource for algorithms
grouped by problems.)

\paragraph{Run a few numbers.}
Do a few tests with small programs and evaluate their scaling
performance. Don't just look at the actual numbers themselves---also
consider the scaling behavior as you vary the problem size.  If the
program does not exhibit the scaling behavior you expect
theoretically, it has a bug. If so, fix the bug before proceeding! (In
general, algorithms follow the theoretical scaling prediction quite
closely for all but the smallest of problem sizes.) Extrapolate to
real-sized problems: can you live with the expected runtime predictions?

\paragraph{Forget standard software engineering practices.}
It is a standard assumption in current software engineering that
developer time is the scarcest resource and that programs should be
designed and implemented accordingly. Computationally intensive
programs are one case~where this is not true: if you are likely to max
out the machine, then it's worth having~the developer---rather than
the computer---go the extra mile.  Additional developer time may very
well make the difference between an ``infeasible'' problem and a
solved one.


For instance, in situations where you are pressed for space, it might
very well make sense to write your own container implementations
instead of relying on the system-provided hash map.  Beware of the
trap of conditioned thinking, though: in one project I worked on, we
knew that we would have a memory\vadjust{\pagebreak} size problem and that we therefore
had to keep the size of individual elements small. On the other hand,
it was not clear at first whether the 4-byte Java \texttt{int} data
type would be sufficient to represent all required values or whether
we would have to use the 8-bye Java \texttt{long} type.  In response,
someone suggested that we \emph{wrap} the atomic data type in an
object so we could swap out the implementation, in case the 4-byte
\texttt{int} turned out to be insufficient.  That's a fine approach in
a standard software engineering scenario (``encapsulation'' and all
that), but in this situation---where space was at a premium---it
missed the point entirely: the space that the Java wrapper would have
consumed (in addition to its data members) would have been larger than
the payload!

Remember: standard software engineering practices are typically
intended to trade machine resources for developer resources.  However,
for computationally intensive problems, machine resources (not
developer time) are the limiting factor.

\paragraph{Don't assume that parallelization will be
possible.}\index{parallelization}  Don't assume that you'll be
able to partition the problem in such a way that simultaneous execution on
multiple machines (\ie, parallelization) will be possible, until you have
developed an actual, concrete, implementable algorithm---many computational
problems don't parallelize well. Even if you can come up with a parallel
algorithm, performance may be disappointing: hidden costs (such as
communication overhead) often lead to performance that is much poorer than
predicted; a cluster consisting of twice as many nodes often exhibits
a behavior \emph{much} less than double the original one! Running
realistic tests (on realistically sized data sets and on realistically
sized clusters) is harder for parallel programs than for single
processor implementations---but even more important.

\paragraph{Leave yourself some margin.}
Assume that the problem size will be larger by a factor of 3 and that
hardware will deliver only 50 percent of theoretically predicted
performance.

\paragraph{If the results are not wholly reassuring, explore
alternatives.}\vspace*{3pt} Take the results for the expected runtime and
memory requirements that you obtained from theoretical predictions and the
tests that you have performed seriously. Unless you seem able to meet your
required benchmarks \emph{comfortably}, explore alternatives.  Consider better
algorithms, research whether the problem can be simplified or whether
the problem can be approached in an entirely different manner, and
look into approximate or heuristic solutions.  If you feel yourself
stuck, get help!

\paragraph{If you can't make it work on paper, \emph{STOP}.}
It won't work in practice, either.  It is a surprisingly common
anti-pattern to see the warning signs early but to press on regardless
with the hopeful optimism that ``things will work themselves out
during implementation.'' This is entirely misguided: nothing will work
out better as you proceed with an implementation; everything is always
a bit worse than expected.\vfill\pagebreak

Unless you can make it work on paper and make it work
\emph{comfortably}, there is no point in proceeding!

The recurring recommendation here is that nobody is helped by a
project that ultimately fails, because it was impossible (or at least
infeasible) from the get-go. Unless you can demonstrate at least the
feasibility of a solution (at an acceptable price point!), there is no
use to proceed. And everybody is much better off knowing this ahead of
time.

\index{performance!matrix operations and other computational applications|)}

\vspace*{-6pt}
% ============================================================
\section{What About Map/Reduce?}

\index{map/reduce techniques} 

Won't the map/reduce family of techniques make most of these
considerations obsolete? The answer, in general, is \emph{no}.

It is important to understand that map/reduce is not actually a clever
algorithm or even an algorithm at all. It is a piece of
\emph{infrastructure} that makes naive algorithms convenient.

That's a whole different ball game. The map/reduce approach does not
speed up any particular algorithm at all. Instead, it makes the
parallel execution of many subproblems convenient. For map/reduce to
be applicable, therefore, it must be possible to \emph{partition} the
problem in such a way that individual partitions don't need to talk to
each other. Search is such an application that is trivially
parallelizable, and many (if not all) successful current applications
of map/reduce that I am aware of seem to be related to generalized forms
of search.

This is not to say that map/reduce is not a very important advance.
(Any device that makes an existing technique orders of magnitudes more
convenient is an important innovation!) At the moment, however, we are
still in the process of figuring how which problems are most amenable
to the map/reduce approach and how best to adapt them.  I suspect that
the algorithms that will work best on map/reduce will \emph{not} be
straightforward generalizations of serial algorithms but instead will
be algorithms that would be entirely unattractive on a serial
computer.

It is also worth remembering that parallel computation is not new.
What has killed it in the past was the need for different partitions
of the problem to communicate with each other: very quickly, the
associated communication overhead annihilated the benefit from
parallelization. This problem has not gone away, it is merely masked
by the current emphasis on search and searchlike problems, which allow
trivial parallelization without any need for communication among
partitions. I worry that more strictly computational applications
(such as the matrix multiplication problem discussed earlier or the
simulation of large physical systems) will require so much sharing of
information among nodes that the map/reduce approach will appear
unattractive.

Finally, amid the excitement currently generated by map/reduce, it
should not be forgotten that its total cost of ownership (including
the long-term \emph{operational} cost of maintaining the required
clusters as well as the associated network and storage infrastructure)
is not yet known.\vadjust{\pagebreak}  Although map/reduce installations make distributed
computing ``freely'' available to the individual programmer, the
required hardware installations and their operations are anything but
``free.''

In the end, I expect map/reduce to have an effect similar to the one
that compilers had when they came out. The code that they produced was
less efficient than handcoded assembler code, but the overall efficiency
gain far outweighed this local disadvantage.

But keep in mind that even the best compilers have rendered neither
Quicksort nor indexed lookup obsolete.

% ============================================================
\section{Workshop: Generating Permutations}

\index{permutations} 

Sometimes, you have to see it to believe it. In this spirit, let's
write a program that calculates all permutations (\ie, all possible
rearrangements) of a set. (That is, if the set is \texttt{[1,2,3]},
then the program will generate \texttt{[1,2,3]}, \texttt{[1,3,2]},
\texttt{[2,1,3]}, \texttt{[2,3,1]}, \texttt{[3,1,2]},
\texttt{[3,2,1]}.)  You can imagine this routine to be part of a
larger program: in order to solve the Traveling Salesman problem
exactly, for example, one needs to generate all possible trips (\ie,
all permutations of the cities to visit) and evaluate the associated
distances.

Of course, we all ``know'' that the number of permutations grows as
$n! = 1 \cdot 2 \cdot 3 \dotsb n$, where $n$ is the number of elements
in the set and that the factorial function grows ``quickly.''
Nevertheless, you have to see it to believe it. (Even I was shocked by
what I found when developing and running the program below!)

The program that follows reads a positive integer \texttt{n} from the
command line and then generates all permutations of a list of
\texttt{n} elements, using a recursive algorithm. (It successively
removes one element of the list, generates all permutations of the
remainder, and then tacks the removed element back on to the results.)
The time required is measured and printed.

\begin{verbatim}
import sys, time

def permutations( v ):
    if len(v) == 1: return [ [v[0]] ]

    res = []
    for i in range( 0, len(v) ):
        w = permutations( v[:i] + v[i+1:] )
        for k in w:
            k.append( v[i] )
        res += w
        
    return res


n = int(sys.argv[1])
v = range(n)
\end{verbatim}
\begin{verbatim}

t0 = time.clock()
z = permutations( v )
t1 = time.clock();

print n, t1-t0
\end{verbatim}

(You may object to the use of recursion here, pointing out that Python
does not allow infinite depth of recursion. This is true but is not a
factor: we will run into trouble long before that constraint comes
into play.)

I highly recommend that you try it. Because we know (or suspect) that
this program might take a while to run when the number of elements is
large, we probably want to start out with three elements. Or with
four.  Then maybe we try five, six, or seven. In all cases, the
program finishes almost \emph{instantaneously}. Then go ahead and run
it with \texttt{n=10}.  Just 10 elements. Go ahead, do it. (But I
suggest you save all files and clean up your login session first, so
you can reboot without losing too much work if you have to.)

Go ahead. You have to \emph{see} it to believe it!\footnote{Anybody
  who scoffs that this example is silly, because ``you should not
  store all the intermediate results; use a generator'' or because
  ``everyone knows you can't find all permutations exhaustively; use
  heuristics'' is absolutely correct---and entirely missing the point.
  I know that this implementation is naive, but---cross your
  heart---would you really have assumed that the naive implementation
  would be in trouble for $n=10$? Especially, when it didn't even
  blink for $n=7$?}

% ============================================================
\section{Further Reading}

% Everybody should have an elementary algorithm book on their shelf ---
% one that is thin and simple enough that it is actually \emph{read}.
% Sedgwick's book fulfills these requirements: the presentation is
% clear and the selection of topics is reasonably comprehensive, but
% the treatment is often not sufficiently deep. Doesn't matter: to
% learn what's out there, this book is good enough, and it provides
% a good starting point for further exploration.

% What about Algos in a Nutshell?

\begin{itemize}
\item \cit{The Algorithm Design Manual}{Steven S.\ Skiena}{2nd ed.,
    Springer}{2008} This is an amazing book, because it presents
  algorithms not as abstract entities to be studied for their own
  beauty but as potential solutions to real problems. Its second half
  consists of a ``hitchhiker's guide to algorithms'': a catalog of
  different algorithms for common problems.  It helps you find an
  appropriate algorithm by asking detailed questions about your
  specific problem and provides pointers to existing implementations.
  In addition, the author's ``war stories'' of past successes and
  failures in the real world provide a vivid reminder that algorithms
  are \emph{real}.
\end{itemize}
