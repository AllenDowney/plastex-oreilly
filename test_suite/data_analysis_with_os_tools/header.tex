
% ============================================================
% O'Reilly Macros

% Alternative item in ``description'' environs, with the headline being
% broken across multiple lines if necesary.

%
\makeatletter
\def\items#1{{\item[]{\leftskip-\leftmargin\hspace*{-6pt}\bfseries#1}}}
\makeatother
%
% “\items{#}” 

% ============================================================
% Special Environments and Commands

% Environ for floated, formal listing w/ caption and label
%    ... this requires the float package: \usepackage{float}
%    ... environ would include a ``verbatim'' environ with actual code
%\newfloat{example}{h}{lst}[chapter]
%\floatname{example}{Example}

% Note that there is also a ``moreverb'' package, which includes
% a ``listing'' environment with line numbers!

\def\maketitle{\par
  \@topnum\z@ % this prevents figures from falling at the top of page 1
  \begingroup
  \@maketitle
  \endgroup
  \c@footnote\z@
  \def\do##1{\let##1\relax}%
  \do\maketitle \do\@maketitle \do\title \do\@xtitle \do\@title
  \do\author \do\@xauthor \do\address \do\@xaddress
  \do\email \do\@xemail \do\curraddr \do\@xcurraddr
  \do\dedicatory \do\@dedicatory \do\thanks \do\thankses
  \do\keywords \do\@keywords \do\subjclass \do\@subjclass
}
\def\contentsname{Contents}
\def\tableofcontents{\@starttoc{toc}\contentsname}

\def\mainmatter{\cleardoublepage\pagenumbering{arabic}}

\newcommand{\cit}[4]{\textit{#1.} #2. #3. #4.\\}
\newcommand{\citnobreak}[4]{\textit{#1.} #2. #3. #4}
\newcommand{\pcit}[6]{``#1.'' #2. \textit{#3} #4 (#5), p.\ #6.}

\newcommand{\tbd}{\par
\begin{center}
\cornersize*{2mm}
\setlength{\fboxsep}{2mm}
\ovalbox{\parbox{8cm}{Coming soon}}
\end{center}
}

\newcommand{\soon}[1]{
\begin{center}
\cornersize*{2mm}
\setlength{\fboxsep}{3mm}
\ovalbox{\parbox{8cm}{Coming soon:\\
\textit{#1}}}
\end{center}
}

\def\@{\spacefactor\@m}

%\def\url{{\itshape }}


% ============================================================
% Special relations

\newcommand{\beh}{\cong}            % for: behaves as
\newcommand{\scl}{\sim}             %      scales as
\newcommand{\app}{\approx}          %      approximately equal   
\newcommand{\so}{\Longrightarrow}
\newcommand{\der}{\partial}
\newcommand{\defeq}{\stackrel{\scriptscriptstyle \text{def}}{=} }
\newcommand{\askeq}{\stackrel{\text{?}}{=} }


% ============================================================
% Balanced delimiters

\newcommand{\paren}[1]{\ensuremath{\left( #1 \right)}}
\newcommand{\brackets}[1]{\ensuremath{\left[ #1 \right]}}
\newcommand{\braces}[1]{\ensuremath{\left\{ #1 \right\} }}
\newcommand{\abs}[1]{\ensuremath{\lvert #1 \rvert}}
\newcommand{\norm}[1]{\ensuremath{\lVert #1 \rVert}}
\newcommand{\angles}[1]{\ensuremath{\left\langle #1 \right\rangle}}


% ============================================================
% Derivatives and Integrals

% Usage: \diff{f}{x}  OR  \diff[3]{f}{x}

% The optional arg, which is the order of the diff, is optional!

% Differential quotient and operator
\newcommand{\diff}[3][]{\mathop{}\!\frac{\text{d}^{#1} #2}{\text{d} #3^{#1} }}
\newcommand{\Diff}[2][]{\mathop{}\!\frac{\text{d}^{#1} }{\text{d} #2^{#1} }}

% Partial differential quotient and operator
\newcommand{\pdif}[3][]{\mathop{}\!\frac{\partial^{#1} #2}{\partial #3^{#1} }}
\newcommand{\pDif}[2][]{\mathop{}\!\frac{\partial^{#1} }{\partial #2^{#1} }}

% According to TUGboat 2,29 (2008):
% \newcommand{\ud}{\mathop{}\!\mathrm{d}} puts space in front of diff op

% ---

% Right Measure
\newcommand{\rms}[1][]{\mathop{}\!\text{d}#1}

% Left Measure
\newcommand{\lms}[1][]{\!\text{d}#1}


% ============================================================
% Marginal notes, etc

\newcommand{\warn}{
  \setlength{\marginparsep}{-0.0in}
  \marginpar{\center{ \fbox{ ??? } }}
}

% \newcommand{\note}[1]{}
\renewcommand{\note}[1]{
  \setlength{\marginparsep}{0.0in}
  \marginpar{\tiny #1}
}

\newcommand{\textnote}[1]{%
  \begin{center}
    \setlength{\fboxrule}{3\fboxrule}
    \framebox[0.6\textwidth]{#1}
  \end{center}
}


% ============================================================
% Text styles
\newcommand{\etc}{etc}
\newcommand{\eg}{\emph{e.g.}}
\newcommand{\ie}{\emph{i.e.}}
\newcommand{\cf}{c.f.\ }
\newcommand{\etal}{\emph{et. al.}}

