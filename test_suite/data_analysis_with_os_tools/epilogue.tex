
\chapter{Epilogue: Facts Are Not Reality}{}{}
\label{ch:epilogue}

\Fint{The last (not least) important skill when working with data is to
keep in mind that data is only} part of the picture. In particular, when
one is working intensely with data oneself, it is all too easy to forget
that just about everyone else will have a different
perspective.\vspace*{-8pt}


\leaflong{-12pt}
%% \vspace*{1em}
%\begin{unnumlist}
\paragraph{When the data contradicts appearances, appearances will win.}
 \index{decision making, just the facts}  Almost
always, at least. Abstract ``data'' will have little or no credibility
when compared with direct, immediate observation. This has been one of
my most common experiences. A manager observes a pile of defective
items---and no amount of ``data'' will convince him that avoiding those
defects will cost more than the defects themselves. A group of workers
spends an enormous amount of effort on some task---and no amount of
``data'' will convince them that their efforts make no measurable
difference to the quality of the product.
%\end{unnumlist}

If something strongly \emph{appears} to be one way, then it will be
very, very difficult to challenge that appearance based on some
abstract analysis---no matter how ``hard'' your facts may be.

And it can get ugly. If your case is watertight, so that your analysis
cannot be refuted, then you may next find that your \emph{personal}
credibility or integrity is being challenged.

Never underestimate the persuasive power of appearance.\vspace*{-8pt}


% \vspace*{1em}
%\begin{unnumlist}
\paragraph{Data-driven decision making is a contradiction in terms.} 
\index{data-driven decision making}  Making a decision means that
someone must stick his or her neck out and
\emph{decide}. If we wait until the situation is clear or let ``the
data'' dictate what we do, then there is no longer any decision
involved. This also means  that if things don't turn out well, then
nobody will accept the blame (or the responsibility) for the outcome:
after all, we did what ``the data'' told us to do.
%\end{unnumlist}

It is a fine line. Gut-level decisions can be annoyingly random (this
way today, that way tomorrow). They can also lead to a lack of
accountability: ``It was my decision to do X that led to
Y!''---without a confirming look at some data, who can say?

Studying data can help us understand the situation in more detail and
therefore make better-informed decisions. On the other hand, data can
be misleading in subtle ways. For instance, by focusing on ``data'' it
is easy to overlook aspects that are important but for which no data
is available (including but not limited to ``soft factors''). Also,
keep in mind that data is always \emph{backward} looking: there is no
data available to evaluate any truly novel idea!

Looking at data can help illuminate the situation and thereby help us
make better decisions. But it should not be used to absolve everyone
from taking individual responsibility.


% \vspace*{1em}
%\begin{unnumlist}
\paragraph{Sometimes the only reason you need is that it is the right
thing to do.} Some organizations feel as if you would not
put out a fire in the mail room, unless you first ran a controlled
experiment and developed a business case for the various alternatives.
Such an environment can become frustrating and stifling; if the same
approach is being applied to human factors such as creature comforts
(better chairs, larger monitors) or customer service (``sales don't dip
proportionally if we lower the quality of our product''), then it can
start to feel toxic pretty quickly.
%\end{unnumlist}

Don't let ``data'' get in the way of ethical decisions. \index{ethics} 


% \vspace*{1em}
%\begin{unnumlist}
\paragraph{The most important things in life can't be
measured.}  It is a fallacy to believe that, just because
something can't be measured, it doesn't matter or doesn't even exist. 
And a pretty tragic fallacy at that.
%\end{unnumlist}

% The belief that because something can't be measured it doesn't matter,
% or does not even exist, is a fallacy. And a pretty tragic one at that.

\clearpage

\thispagestyle{empty}

\cleardoublepage