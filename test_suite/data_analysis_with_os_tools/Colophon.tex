\documentclass{Oreilly5980006}
%\documentclass[draft]{Oreilly5980006}

\voffset-.5in
\hoffset-.5in

\usepackage{amsfonts}
\usepackage{amsmath,amssymb}
\usepackage{graphicx}
\usepackage{fancybox}
\usepackage{url}

\makeatletter
\def\@makeschapterhead#1{%
  \vbox to 68pt{\vskip-\headheight\vskip\headsep\vskip-\topskip\parindent\z@ 
{\hskip256pt{\COAfont  #1}\par}%
    \vfill
   }}
\makeatother 

% ============================================================
% O'Reilly Macros

% Alternative item in ``description'' environs, with the headline being
% broken across multiple lines if necesary.

%
\makeatletter
\def\items#1{{\item[]{\leftskip-\leftmargin\hspace*{-6pt}\bfseries#1}}}
\makeatother
%
% “\items{#}” 

% ============================================================
% Special Environments and Commands

% Environ for floated, formal listing w/ caption and label
%    ... this requires the float package: \usepackage{float}
%    ... environ would include a ``verbatim'' environ with actual code
%\newfloat{example}{h}{lst}[chapter]
%\floatname{example}{Example}

% Note that there is also a ``moreverb'' package, which includes
% a ``listing'' environment with line numbers!

\def\maketitle{\par
  \@topnum\z@ % this prevents figures from falling at the top of page 1
  \begingroup
  \@maketitle
  \endgroup
  \c@footnote\z@
  \def\do##1{\let##1\relax}%
  \do\maketitle \do\@maketitle \do\title \do\@xtitle \do\@title
  \do\author \do\@xauthor \do\address \do\@xaddress
  \do\email \do\@xemail \do\curraddr \do\@xcurraddr
  \do\dedicatory \do\@dedicatory \do\thanks \do\thankses
  \do\keywords \do\@keywords \do\subjclass \do\@subjclass
}
\def\contentsname{Contents}
\def\tableofcontents{\@starttoc{toc}\contentsname}

\def\mainmatter{\cleardoublepage\pagenumbering{arabic}}

\newcommand{\cit}[4]{\textit{#1.} #2. #3. #4.\\}
\newcommand{\citnobreak}[4]{\textit{#1.} #2. #3. #4}
\newcommand{\pcit}[6]{``#1.'' #2. \textit{#3} #4 (#5), p.\ #6.}

\newcommand{\tbd}{\par
\begin{center}
\cornersize*{2mm}
\setlength{\fboxsep}{2mm}
\ovalbox{\parbox{8cm}{Coming soon}}
\end{center}
}

\newcommand{\soon}[1]{
\begin{center}
\cornersize*{2mm}
\setlength{\fboxsep}{3mm}
\ovalbox{\parbox{8cm}{Coming soon:\\
\textit{#1}}}
\end{center}
}

\def\@{\spacefactor\@m}

%\def\url{{\itshape }}


% ============================================================
% Special relations

\newcommand{\beh}{\cong}            % for: behaves as
\newcommand{\scl}{\sim}             %      scales as
\newcommand{\app}{\approx}          %      approximately equal   
\newcommand{\so}{\Longrightarrow}
\newcommand{\der}{\partial}
\newcommand{\defeq}{\stackrel{\scriptscriptstyle \text{def}}{=} }
\newcommand{\askeq}{\stackrel{\text{?}}{=} }


% ============================================================
% Balanced delimiters

\newcommand{\paren}[1]{\ensuremath{\left( #1 \right)}}
\newcommand{\brackets}[1]{\ensuremath{\left[ #1 \right]}}
\newcommand{\braces}[1]{\ensuremath{\left\{ #1 \right\} }}
\newcommand{\abs}[1]{\ensuremath{\lvert #1 \rvert}}
\newcommand{\norm}[1]{\ensuremath{\lVert #1 \rVert}}
\newcommand{\angles}[1]{\ensuremath{\left\langle #1 \right\rangle}}


% ============================================================
% Derivatives and Integrals

% Usage: \diff{f}{x}  OR  \diff[3]{f}{x}

% The optional arg, which is the order of the diff, is optional!

% Differential quotient and operator
\newcommand{\diff}[3][]{\mathop{}\!\frac{\text{d}^{#1} #2}{\text{d} #3^{#1} }}
\newcommand{\Diff}[2][]{\mathop{}\!\frac{\text{d}^{#1} }{\text{d} #2^{#1} }}

% Partial differential quotient and operator
\newcommand{\pdif}[3][]{\mathop{}\!\frac{\partial^{#1} #2}{\partial #3^{#1} }}
\newcommand{\pDif}[2][]{\mathop{}\!\frac{\partial^{#1} }{\partial #2^{#1} }}

% According to TUGboat 2,29 (2008):
% \newcommand{\ud}{\mathop{}\!\mathrm{d}} puts space in front of diff op

% ---

% Right Measure
\newcommand{\rms}[1][]{\mathop{}\!\text{d}#1}

% Left Measure
\newcommand{\lms}[1][]{\!\text{d}#1}


% ============================================================
% Marginal notes, etc

\newcommand{\warn}{
  \setlength{\marginparsep}{-0.0in}
  \marginpar{\center{ \fbox{ ??? } }}
}

% \newcommand{\note}[1]{}
\newcommand{\note}[1]{
  \setlength{\marginparsep}{0.0in}
  \marginpar{\tiny #1}
}

\newcommand{\textnote}[1]{%
  \begin{center}
    \setlength{\fboxrule}{3\fboxrule}
    \framebox[0.6\textwidth]{#1}
  \end{center}
}


% ============================================================
% Text styles
\newcommand{\etc}{etc}
\newcommand{\eg}{\emph{e.g.}}
\newcommand{\ie}{\emph{i.e.}}
\newcommand{\cf}{c.f.\ }
\newcommand{\etal}{\emph{et. al.}}




\begin{document}



\chapter*{About the Author}{}{}
\thispagestyle{empty}

After previous careers in physics and software 
development, {\bf Philipp K. Janert} currently 
provides consulting services for data analysis, 
algorithm development, and mathematical modeling. 
He has worked for small start-ups and in large 
corporate environments, both in the U.S. and 
overseas. He prefers simple solutions that work 
to complicated ones that don't, and thinks that 
purpose is more important than process. Philipp 
is the author of ``Gnuplot in Action:  Understanding
Data with Graphs'' (Manning Publications), and has 
written for the O'Reilly Network, IBM developerWorks, 
and IEEE Software. He is named inventor on a handful 
of patents, and is an occasional contributor to CPAN. 
He holds a Ph.D. in theoretical physics from the 
University of Washington. Visit his company website 
at \url{www.principal-value.com.}


%{\bf Philipp K. Janert}
%is chief consultant at Principal Value, LLC. He has worked for small start-ups and in large corporate environments---including
%Amazon.com---where he initiated and led several projects to improve order fulfillment processes. Philipp has written about software and
%software development for the O'Reilly Network, IBM developerWorks, IEEE Software, and {\it Linux Magazine}.\vspace*{-6pt}

%\leaflong{6pt}

\section*{Colophon}
The animal on the cover of {\it Data Analysis with Open Source Tools} is a common kite, most likely a member of the genus {\it Milvus}. Kites are
medium-size raptors with long wings and forked tails. They are noted for their elegant, soaring flight. They are also called ``gledes''
(for their gliding motion) and, like the flying toys, they appear to ride effortlessly on air currents.

The genus {\it Milvus} is a group of Old World kites, including three or four species and numerous subspecies. These kites are opportunistic
feeders that hunt small animals, such as birds, fish, rodents, and earthworms, and also eat carrion, including sheep and cow carcasses.
They have been observed to steal prey from other birds. They may live 25 to 30~years in the wild.

The genus dates to prehistoric times; an Israeli {\it Milvus pygmaeus} specimen is thought to be between 1.8 million and 780,000 years old.
Biblical references to kites probably refer to birds of this genus. In {\it Coriolanus}, Shakespeare calls Rome ``the city of kites and crows,''
commenting on the birds' prevalence in urban areas.

The most widespread member of the genus is the black kite ({\it Milvus migrans}), found in Europe, Asia, Africa, and Australia. These kites are
very common in many parts of their habitat and are well adapted to city life. Attracted by smoke, they sometimes hunt by capturing small
animals fleeing from fires.

The other notable member of {\it Milvus} is the red kite ({\it Milvus milvus}), which is slightly larger than the black kite and is distinguished by a
rufous body and tail. Red kites are found only in Europe. They were very common in Britain until 1800,\vadjust{\pagebreak} but the population was devastated
by poisoning and habitat loss, and by 1930, fewer than 20 birds remained. Since then, kites have made a comeback in Wales and have been
reintroduced elsewhere in Britain.

\thispagestyle{empty}


The cover image is from Cassell's {\it Natural History}, Volume III. The cover font is Adobe ITC Garamond; the text font is Adobe's Meridien-Roman; the
heading font is Adobe Myriad Condensed; and the code font is LucasFont's TheSansMonoCondensed.

\clearpage

\thispagestyle{empty}

\cleardoublepage
\end{document}
